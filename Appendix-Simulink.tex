\chapter{On Simulink}\label{App:Simulink}

\section{Model Linearizer Tools}\label{App:Simulink:ModelLinearizer}
The model linearizer tools is an ``App'' in Simulink that allows an engineer to
(1) linearize a system and (2) perform linear system analyses such as Bode
plots and Step Responses. We won't be needing the first feature, since our
systems are already linear. This section is a brief tutorial on how to use
the tools to acquire step responses, Bode plots and Nyquist plots.

\subsection{The Model Linearizer App}
To open the Model Linearizer App, look to the top of your Simulink Model
window and click the ``\texttt{APPS}'' button like so
%
\begin{center}
  \includegraphics[width=0.8\linewidth]{%
    images/app-model-linearizer-open-1.png%
  }
\end{center}
%
Then proceed to opening the application by clicking the ``\texttt{Model
Linearizer}'' button,
%
\begin{center}
  \includegraphics[width=0.8\linewidth]{%
    images/app-model-linearizer-open-2a.png%
  }
\end{center}
%
If you cannot find it located in the toolbar depicted above, then press the
dropdown on the far right of the toolbar,
%
\begin{center}
  \includegraphics[width=0.8\linewidth]{%
    images/app-model-linearizer-open-2b.png%
  }
\end{center}
%
If you still cannot find it, you must not have the \texttt{Control Systems
Toolbox} installed. Ensure you've installed the toolboxes listed in the
Introduction. Upon opening the Model Linearizer app, you
will have a window that looks like Figure~\ref{fig:app1:model-linearizer}.
%
\begin{figure}[H]
  \centering
  \includegraphics[width=0.8\linewidth]{%
    images/app-model-linearizer-opened.png%
  }
  \caption[The Model Linearizer App]{%
    What the Model Linearizer application looks like when you first open it.%
  }
  \label{fig:app1:model-linearizer}
\end{figure}

\subsection{Setting up your System: Input to Output}
\label{App:Simulink:ModelLinearizer:2}
Now that you know how to open up the Model Linearizer, we must now prepare
the system we want to analyze. The Model Linearizer tool doesn't know what
we consider an ``input'' and what we consider an ``output'' just from our
Simulink diagram. But you do! Whenever you want to take a step response, you
know where you input the step and what measurements you want to observe.
The goal of the following steps is to show you how to indicate to the
Model Linearizer this fact.

First pull up the Simulink model of concern. I'll use the Lab 1 diagram as
an example,
%
\begin{center}
  \includegraphics[width=0.8\linewidth]{images/app-model-linearizer-io-1.png}
\end{center}
%
Suppose we wanted to analyze the relationship between the input to the gain
to the output after the plant \(P(s).\) Right click the signal prior to the
gain and you will have a menu like that of Figure~\ref{fig:app1:io-menu:a}.
%
\begin{figure}
  \centering
  \includegraphics[width=0.8\linewidth]{%
    images/app-model-linearizer-io-2a.png%
  }
  \caption{Signal context menu and Linear Analysis points submenu.}
  \label{fig:app1:io-menu:a}
\end{figure}
%
\begin{figure}
  \centering
  \includegraphics[width=0.8\linewidth]{%
    images/app-model-linearizer-io-2b.png%
  }
  \caption{The Linear Analysis context submenu.}
  \label{fig:app1:io-menu:b}
\end{figure}
%
Since we would like to indicate to the Model Linearizer that the input to the
gain is an input signal, we select ``\texttt{Input Perturbation}.'' We repeat
the \emph{same} process for the output
of \(P(s)\) except we select ``\texttt{Output Measurement}.'' The result is
depicted in Figure~\ref{fig:app1:io-signals}. Note the little annotations
above the signal. A ``down'' arrow denotes an input and an ``up'' arrow denotes
an output signal.
%
\begin{figure}
  \centering
  \includegraphics[width=0.8\linewidth]{images/app-model-linearizer-io-3.png}
  \caption{%
    The result after setting up an input signal and output
    measurement signal.%
  }
  \label{fig:app1:io-signals}
\end{figure}
%
You will have to repeat this process whenever you want to change where you
apply your input or when you want to observe a different output. Note that
the choice of ``\texttt{Input Perturbation}'' and
``\texttt{Output Measurement}'' is not arbitrary; it is intentional! This
allows us to quickly change the configuration of the system (close the loop)
and get closed loop measurements as well without having to change the
what the input is!

\FloatBarrier
\subsection{Acquiring Plots}
\label{App:Simulink:ModelLinearizer:3}
Having indicated to the Model Linearizer app the input and output signal, you
are now ready to acquire a step response. Open the Model Linearizer. Look at
the top bar of the Model Linearizer and confirm that it looks like so
%
\begin{center}
  \includegraphics[width=0.8\linewidth]{%
    images/app-model-linearizer-toolbar.png%
  }
\end{center}
%
Try pressing the ``\texttt{Step Response}'' button! You should get a response
like that of Figure~\ref{fig:app1:stepresponse}. Great! Now try clicking
on a point of the curve. You will then get something we call a \textbf{data
tip} or \textbf{cursor}, shown in Figure~\ref{fig:app1:cursors}.
You can drag this cursor (the black dot) around to a specific point on the
curve, allowing you to get accurate readings. Use this feature!
%
Similarly, one can capture a Bode plot, as depicted in Figure
\ref{fig:app1:bodeplot}, by pressing the ``\texttt{Bode}'' button.

Now that we can acquire these plots, how do we \emph{save} them! Notice that
above every plot is a name. For example, Figure~\ref{fig:app1:bodeplot} is
in a tab named ``Bode Plot 1.'' With this selected, you will see in the top
bar a larger tab with the same name. This is depicted in Figure
\ref{fig:app1:capturefigure}. In that tab, you can then press
``\texttt{Print to Figure}.'' Upon doing so, you have a Matlab Figure pop-up,
which then gives you the option --- under \texttt{File/Save As} --- to
export a \texttt{.png} or other image file.
%
\begin{figure}
  \centering
  \includegraphics[width=0.8\linewidth]{%
    images/app-model-linearizer-stepresponse.png%
  }
  \caption[Capturing a Step Response in the Model Linearizer App]{%
    Capturing a step response in the Model Linearizer app.
  }
  \label{fig:app1:stepresponse}
\end{figure}
%
\begin{figure}
  \centering
  \includegraphics[width=0.8\linewidth]{%
    images/app-model-linearizer-cursors.png%
  }
  \caption[Data Tips in Matlab Figures]{%
    I put two data tips, one near the settling value and another at
    just over \(0.5\) in amplitude.
  }
  \label{fig:app1:cursors}
\end{figure}
%
\begin{figure}
  \centering
  \includegraphics[width=0.8\linewidth]{%
    images/app-model-linearizer-bodeplot.png%
  }
  \caption[Bode Plot captured in the Model Linearizer App]{%
    A bode plot captured in the Model Linearize app. Do you know what special
    points I've marked on the Bode plot? Name them. Why are they important?
  }
  \label{fig:app1:bodeplot}
\end{figure}
%
\begin{figure}
  \centering
  \includegraphics[width=0.8\linewidth]{%
    images/app-model-linearizer-capturefigure.png%
  }
  \caption[Acquiring a Figure in the Model Linearizer App]{%
    This screenshot depicts how to turn a data acquisition into a Matlab
    Figure.
  }
  \label{fig:app1:capturefigure}
\end{figure}
%
