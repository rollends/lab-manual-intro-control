\chapter{Bringing it Together: Lateral Motion Control}\label{Lab:3}
Imagine you are a young, aspiring, junior engineer at VenX, a private company in an alternate Universe, manufucturing a mining probe on Venus.
\footnote{This society isn't particularly concerned about the environmental impact. It isn't our planet, right?}
Because of your junior status, they have only asked you to design a simple controller that ensures the lunar module lands at the correct location.
The controller you design controls the level of horizontal thrust to achieve a desired position relative to some point on the surface.
You may assume that all units are in standard units.
Distances are measured in meters, time is measured in seconds, mass is measured in kilograms and all other units are such that they are consistent with that.

\section{Objectives}
\begin{enumerate}[label=(\arabic*)]
  \item{
    \textbf{Use} the skills you learned in Labs~\ref{Lab:1} and~\ref{Lab:2}
    to identify parameters of an opaque, black box system.
  }
  \item{
    \textbf{Design} a proportional inner-loop controller that stabilizes the velocity and controls it to a desired value.
  }
  \item{
    \textbf{Design} a proportional outer-loop controller that stabilizes the relative position of the lunar module with respect to a set point on the ground.
  }
  \item{
    \textbf{Derive} a mathematical expression for your controller and use this to \textbf{achieve} a desired control objective.
  }
  \item{
    \textbf{Compare and contrast} your stabilizing control design with that of Lab~\ref{Lab:2}.
  }
  \item{
    \textbf{Appreciate} how your controller deals with disturbances and unmodelled dynamics.
  }
\end{enumerate}

\section{Experimental Procedure}\label{Lab:3:Experiment}
Unlike previous labs, you will now work with two Simulink models.
Moreover, this lab will involve a larger number of moving parts;
you will introduce three feedback interconnections!

In Section~\ref{Lab:3:Part:I} you will stabilize the physical plant that maps input forces to an output horizontal velocity.
You will treat velocity as the measurement (output).
You can then design a controller

\subsection{Part I: Stabilize and Identify the Plant}\label{Lab:3:Part:I}
Your deliverables for this subsection are
%
\begin{deliverable}[label={del:lab3:g1:1}]
  \textbf{Choose} and \textbf{record} a gain value \(K_1\) for your stabilizing gain.
\end{deliverable}
%
\begin{deliverable}[label={del:lab3:g1:2}]
   \textbf{Acquire} the step response from the signal labelled \(w(t)\) to the signal \(v(t)\) when the signal \(v(t)\) is connected to the \(G_1\) loop summing junction (aptly named ``\(G_1\) Loop'').
\end{deliverable}
%
\begin{deliverable}[label={del:lab3:g1:3}]
  \textbf{Measure} the time constant \(\tau_1\) of your stabilized plant using the step response acquired in Deliverable~\ref{del:lab3:g1:2}.
\end{deliverable}
%
This section concerns the Part I area of the Simulink model
\begin{center}
  \texttt{Lab\_3\_Velocity\_Controlled\_Landing\_Module.slx}.
\end{center}
In this lab, your plant is a simulated physical system.
The system has nonlinear dynamics, but we will treat it as a linear system and you will observe the power of linear control.
First, in order to even design a controller to achieve our objectives, we must stabilize the system.
The reason for this is quite simple. A physical system normally integrates the input force into velocity (and then position) and, unless there are external forces such as friction, the plant is of the form
\[
  P(s)
    =
      \frac{1}{M s}
\]
where \(M > 0\) is the mass of the landing module.
The step response of this system is unbounded.
It is true that there is friction in your model but it is often hard to model friction in reality, so we will take the above equation as our model.
To stabilize the plant, consider closing the loop in the following way
%
\begin{center}
  \begin{tikzpicture}[x=1in, y=1in]

    \node [draw, smooth_block] (Plant) {\(P(s)\)};
    \node [draw, smooth_block, left = 0.75 of Plant] (Gain1) {\(K_1\)};
    \node [draw, smooth_sum, left = 0.75 of Gain1] (Sum1) {};
    \node [right = 0.75 of Plant] (after_plant) {};
    \node [right = 0.75 of after_plant] (v) {};
    \node [left = 0.75 of Sum1] (w) {};

    \node [smooth_annotate, below = 0 of Gain1] {Stabilizing Gain};
    \node [smooth_annotate, below = 0 of Plant] {Plant};

    \draw [arrow, smooth_path]
      (Plant.east) -- (after_plant.base) -- (v.base);
    \draw [arrow, smooth_path]
      (Plant.east)
      --
      (after_plant.base)
      --
      +(0, -0.75)
      -|
      (Sum1.south)
      node [below right] {\(-\)};
    \draw [arrow, smooth_path]
      (Sum1.east)
      --
      (Gain1.west);
    \draw [arrow, smooth_path]
      (Gain1.east)
      --
      (Plant.west);
    \draw [arrow, smooth_path]
      (w.base)
      --
      (Sum1.west);
  \end{tikzpicture}
\end{center}
%
with a randomly chosen gain \(K_1 > 0.\)
The gain you choose is up to you but make sure you record it.
\begin{procedure}[label={proc:lab3:stabilize}]
  This procedure simply reiterates the steps above.
  \begin{enumerate}[label={(\arabic*)}]
    \item{%
      \textbf{Close} the loop as described above.
      You should connect the signal labelled \(v\) to the summing junction labelled ``\(G_1\) Loop.''%
    }
    \item{%
      \textbf{Choose} and \textbf{set} a random positive gain for \(K_1.\)
      The block is named ``Stabilizing Gain'' in the Simulink model.%
    }
    \item{%
      \textbf{Acquire} a step response from the signal \(w(t)\) to the signal \(v(t).\)
      Remember from Lab 2 that you need to set \(w\) as an input perturbation and \(v\) as an output measurement.
    }
    \item{%
      \textbf{Measure} the time constant of this system using the acquired step response. We will call this time constant \(\tau_1.\)
    }
  \end{enumerate}
\end{procedure}

\subsection{Part II: Reference Velocity Control Design}\label{Lab:3:Part:II}
When you performed Procedure~\ref{proc:lab3:stabilize}, you will have noticed that the step response was extremely slow.
In particular, you should notice that it took more than a minute of simulated time to achieve a speed of \SI{1}{m/s} (\SI{3.6}{km/h})!
If this is the response speed of our system, we will never be able to control it to a desired position.
In this section, you will solve this problem by designing a proportional error feedback controller that controls to a \emph{desired velocity} but with a much better time constant.
Your deliverables for this section are
%
\begin{deliverable}[label={del:lab3:g2:1}]
  \textbf{Acquire} a (unit) step response from the signal \(v_r(t)\) to the signal \(v(t).\)
  \textbf{Ensure} you have cursors that indicate the following:
  \begin{itemize}
    \item{the time constant is \(\frac{1}{\sqrt{50}} \approx 0.1414.\)}
    \item{the DC gain, or steady-state value, is \(1.\)}
  \end{itemize}
\end{deliverable}
%
This section concerns the Part II area of the Simulink model
\begin{center}
  \texttt{Lab\_3\_Velocity\_Controlled\_Landing\_Module.slx}.
\end{center}
The now stabilized plant, as a transfer function from \(w(t)\) to \(v(t),\) will be referred to as \(G_1(s)\) as is depicted below.
Verify for yourself that
\[
  G_1(s) = \frac{1}{\tau_1 s + 1}
\]
where \(\tau\) is the time constant of \(G_1(s)\) (as estimated in Deliverable~\ref{del:lab3:g1:timeconstant}).
Close the loop around \(G_1(s)\) with a proportional error feedback controller like so
%
\begin{center}
  \begin{tikzpicture}[x=1in, y=1in]

    \node [draw, smooth_block] (Plant) {\(P(s)\)};
    \node [draw, smooth_block, left = 0.25 of Plant] (Gain1) {\(K_1\)};
    \node [draw, smooth_sum, left = 0.25 of Gain1] (Sum1) {};
    \node [right = 0.25 of Plant] (after_plant) {};
    \node [right = 0.50 of after_plant] (v) {};
    \node [left = 0.50 of Sum1] (w) {};

    \draw [arrow, smooth_path]
      (Plant.east) -- (after_plant.base) -- (v.base)
      node [above right] {\(v(t)\)};
    \draw [arrow, smooth_path]
      (Plant.east)
      --
      (after_plant.base)
      --
      +(0, -0.50)
      -|
      (Sum1.south)
      node [below right] {\(-\)};
    \draw [arrow, smooth_path]
      (Sum1.east)
      --
      (Gain1.west);
    \draw [arrow, smooth_path]
      (Gain1.east)
      --
      (Plant.west);
    % \draw [arrow, smooth_path]
    %   (w.base)
    %   --
    %   (Sum1.west);

    \draw [smooth_area_path, fill = Blue, fill opacity = 0.50, text opacity = 1]
      ($(w)+(0.25, 0)$) -- +(0, 0.50) -| node[midway, above, color=Blue] {\(G_1(s)\)} ($(v)+(-0.50,0)$) -- +(0, -0.75) -| ($(w)+(0.25, 0)$);

    \node [draw, smooth_block, left = 0.25 of w] (Gain2) {\(K_2\)};
    \node [draw, smooth_sum, left = 0.50 of Gain2] (Sum2) {};
    \node [left = 0.50 of Sum2] (v_r) {};

    \draw [arrow, smooth_path]
      (Plant.east)
      --
      ($(v)+(-0.25,0)$)
      --
      +(0, -1)
      -|
      (Sum2.south)
      node [below right] {\(-\)};
    \draw [arrow, smooth_path]
      (Sum2.east)
      --
      (Gain2.west);
    \draw [arrow, smooth_path]
      (Gain2.east)
      --
      (Sum1.west);
    \draw [arrow, smooth_path]
      (v_r.base)
      --
      (Sum2.west)
      node [midway, above] {\(v_r(t)\)};

  \end{tikzpicture}
\end{center}
%
Verify that, when the loop is closed, the transfer function from \(v_r(t)\) to \(v(t),\) denoted as \(G_2(s)\) in the Simulink model, takes the form
\[
  G_2(s)
    = \frac{V(s)}{V_r(s)}
    = \frac{K_3 K_2}{1 + K_2} \frac{1}{\frac{\tau_1}{1 + K_2} s + 1}.
\]
%
\begin{procedure}[label={proc:lab3:speedup}]
  In this procedure you use another closed loop to accelerate the response of your system, thereby making it usable to do position step tracking.
  \begin{enumerate}[label={(\arabic*)}]
    \item{%
      \textbf{Close} the loop as described above.
      You should connect the signal labelled \(v\) to the summing junction labelled ``\(G_2\) Loop.''%
    }
    \item{%
      We would like to have the transfer function \(G_2(s)\) to have a time constant of \(\frac{1}{\sqrt{50}}.\)
      This would mean that, for a step change in the \emph{reference velocity}, the \emph{actual velocity} will reach \(63.3\%\) of the reference velocity in \SI{0.1414}{s}.
      Recognize that the time constant of this new closed loop system is
      \[
        \frac{\tau_1}{1 + K_2}
      \]
      where \(\tau_1\) is the time constant of \(G_1(s)\) that you measured earlier.
      As a result, \textbf{solve} for the \(K_2\) you need to have this time constant equal to \(\frac{1}{\sqrt{50}}.\)%
    }
    \item{%
      \textbf{Solve} for the gain \(K_3\) that ensures \(G_2(s)\) has a DC gain of \(1.\)%
    }
    \item{%
      \textbf{Set} the respective blocks corresponding to \(K_2\) and \(K_3\) in the Simulink diagram.%
    }
    \item{%
      \textbf{Acquire} the step response from the signal \(v_r(t)\) to the signal \(v(t).\)
      Remember again to change which signal is the input perturbation from Part I.%
    }
    \item{%
      \textbf{Place} cursors on the time constant and steady-state value to prove you met the specification.%
    }
  \end{enumerate}
\end{procedure}

\subsection{Part III: Reference Position Control Design}\label{Lab:3:Part:III}
Your deliverables for this part are
%
\begin{deliverable}[label={del:lab3:g3:1}]
   \textbf{Acquire} the step response from the signal labelled \(r(t)\) to the signal \(x(t)\) when the signal \(x(t)\) is connected to the \(G_3\) loop summing junction (aptly named ``\(G_3\) Loop'').
   This is the step response of the transfer function \(G_3(s)\) depicted in the Simulink model.
\end{deliverable}
%
\begin{deliverable}[label={del:lab3:g3:2}]
   \textbf{Run} the ``\texttt{visualize\_landing.m}'' script and \textbf{save} both Figures that are generated.
\end{deliverable}
%
This section concerns the Part III area of the Simulink model
\begin{center}
  \texttt{Lab\_3\_Position\_Controlled\_Landing\_Module.slx}.
\end{center}
You will notice that the other Simulink model you worked with is labelled \(G_2(s)\) in this model.
Since you have designed a controller that achieves a desired reference velocity, we leverage it in this model to design an outer loop controller that achieves a desired position.
Close the loop to arrive at a diagram
%
\begin{center}
  \begin{tikzpicture}[x=1in, y=1in]

    \node [draw, smooth_block] (G2) {\(G_2(s)\)};
    \node [draw, smooth_block, right = 0.75 of G2] (Int) {\(\frac{1}{s}\)};
    \node [draw, smooth_block, left = 0.75 of G2] (Gain4) {\(K_4\)};
    \node [draw, smooth_sum, left = 0.75 of Gain4] (Sum3) {};
    \node [right = 0.25 of Int] (after_int) {};
    \node [right = 0.50 of after_int] (x) {};
    \node [left = 0.50 of Sum3] (r) {};

    \draw [arrow, smooth_path]
      (G2.east) -- (Int.west)
      node [above, midway] {\(v(t)\)};
    \draw [arrow, smooth_path]
      (Int.east) -- (after_int.base) -- (x.base)
      node [above] {\(x(t)\)};
    \draw [arrow, smooth_path]
      (Int.east)
      --
      (after_int.base)
      --
      +(0, -0.50)
      -|
      (Sum3.south)
      node [below right] {\(-\)};
    \draw [arrow, smooth_path]
      (Sum3.east)
      --
      (Gain4.west);
    \draw [arrow, smooth_path]
      (Gain4.east)
      --
      (G2.west);
    % \draw [arrow, smooth_path]
    %   (w.base)
    %   --
    %   (Sum1.west);

    \draw [arrow, smooth_path]
      (r.base)
      --
      (Sum3.west)
      node [midway, above] {\(r(t)\)};

  \end{tikzpicture}
\end{center}
%
If you performed Part II correctly, the transfer function \(G_2(s)\) takes the
form
\[
  G_2(s) = \frac{1}{\frac{1}{\sqrt{50}}s + 1}.
\]
Verify that the transfer function from \(r(t)\) to \(x(t),\) denoted \(G_3(s)\) in the Simulink model, equals
\[
  G_3(s) = \frac{K_4\sqrt{50}}{s^2 + \sqrt{50} s + K_4 \sqrt{50}}.
\]
%
\begin{procedure}[label={proc:lab3:regulate}]
  You will make a specific choice for \(K_4\) in this procedure and then simulate the entire non-linear system.
  \begin{enumerate}[label={(\arabic*)}]
    \item{%
      \textbf{Close} the loop as described above.
      You should connect the signal \(x(t)\) to the \(G_3\) loop summing junction (aptly named ``\(G_3\) loop'').%
    }
    \item{%
      \textbf{Determine} and \textbf{set} the gain \(K_4\) so that the system \(G_3(s)\) is critically damped.
      To be clear, with your choice of \(K_4\) you should find that \(\zeta = 1\) when \(G_3(s)\) is put in standard form.%
    }
    \item{%
      \textbf{Acquire} the (unit) step response from the signal \(r(t)\) to the signal \(x(t).\)
      Remember to go into the other model file and remove any annotations you have used there before setting new ones in this model file.%
    }
    \item{%
      \textbf{Run} the ``\texttt{visualize\_landing.m}'' script and \textbf{save} both Figures that are generated.%
      \emph{Do not be alarmed that the behaviour does not look like the step response. You will be expected to address this discrepancy in your report.}
    }
  \end{enumerate}
\end{procedure}
%

\subsection{Report}
