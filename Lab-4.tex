\chapter{PID Analysis}\label{Lab:4}
You are now a slightly more experienced junior engineer at VenX.
This time, you are designing an unmanned aerial vehicle that is dropped into the atmosphere and performs powered glides along predefined paths over the surface.
%
\begin{center}
  %\centering
  \input{Lab_4_UAV_PF.pdf_tex}%
  %\caption{Unmanned Aerial Vehicle (UAV) moving with respect to some line.}
  %\label{fig:lab4:uav}
\end{center}
%
The purpose is to do surveillance\footnote{Don't worry. The CEO of VenX, Lone Dusk, has assured you that the FBI are not using your technology for surveilling citizens in protests. Should you trust him? Probably not. But in this lab, you live in a utopia where people and the government are very honourable.} of the planet's surface to map out the landscape.

The controller you design controls the turning rate (\SI{}{rad/s}) and the output you measure is the distance to the straight line path \(y(t)\) (\SI{}{m}).
In practice the transfer function from this input to \(y(t)\) is highly nonlinear, but the senior engineers have designed an inner loop controller that makes the plant \(P(s)\) look like
\[
  P(s) = \frac{1}{s^2}.
\]
Take \texttt{ECE 486}\footnote{In the robotics course you will learn to just cancel the nonlinearities out.} if you are interested in how to do this.

\section{Objectives}\label{Lab:4:Objectives}
This goals of this lab are to
\begin{enumerate}[label=(\arabic*)]
  \item{
    \textbf{Learn} how the derivative and integral terms of a PID controller affect performance (desireably and undesireably).
  }
  \item{
    \textbf{Tune} a proportional-integral-derivative (PID) controller.
  }
  \item{
    \textbf{Learn} how to use the root locus to assist in control design.
  }
  \item{
    \textbf{Recognize} how zero dynamics can be introduced by controllers and how they can severely impact performance and analysis.
  }
\end{enumerate}
The deliverable dependency graph is
\begin{center}
\begin{tikzpicture}[x=1em, y=1em]
  \node[deliverable] (D1) {Deliverable\\\ref{del:lab4:p1:1}};
  \node[deliverable, below = 2 of D1] (D2) {Deliverable\\\ref{del:lab4:p2:1}--\ref{del:lab4:p2:3}};
  \node[deliverable, below = 2 of D2] (D3) {Deliverable\\\ref{del:lab4:p3:1}--\ref{del:lab4:p3:3}};
  \node[deliverable, right = 2 of D3] (Q1) {Deliverable\\\ref{lab4:report}~\ref{lab4:report:q1}--\ref{lab4:report:q2}};
  \node[deliverable, below = 2 of Q1] (Q4) {Deliverable\\\ref{lab4:report}~\ref{lab4:report:q4}};
  \node[deliverable, right = 2 of Q4] (Q5) {Deliverable\\\ref{lab4:report}~\ref{lab4:report:q5}};
  \node[deliverable, at = {(Q5|-D1)}] (Q3) {Deliverable\\\ref{lab4:report}~\ref{lab4:report:q3}};
  \node[deliverable, right = 2 of Q5] (Q6) {Deliverable\\\ref{lab4:report}~\ref{lab4:report:q6}};
  \node[deliverable, left = 2 of D3] (Q7) {Deliverable\\\ref{lab4:report}~\ref{lab4:report:q7}};

  \draw[signal, arrow] (D1.south) -- (D2.north);
  \draw[signal, arrow] (D2.south) -- (D3.north);
  \draw[signal, arrow] (D3.east) -- (Q1.west);
  \draw[signal, arrow] (D3.south) |- (Q4.west);
  \draw[signal, arrow] (Q4.east) -- (Q5.west);
  \draw[signal, arrow] (Q3.south) -- (Q5.north);
  \draw[signal, arrow] (Q5.east) -- (Q6.west);
  \draw[signal, arrow] (D3.west) -- (Q7.east);

\end{tikzpicture}
\end{center}

\section{Experimental Procedure}\label{Lab:4:Experiment}
This lab is split into three parts.
There is only one Simulink model.
In Part I you will analyze how the derivative term affects the stability of your system.
In Part II you will analyze how the integral term affects the performance of your system.
Finally in Part III you make an incremental change to your PID controller to improve your system's performance qualitatively.
The plant for this lab will be
\[
  P(s) = \frac{1}{s^2}.
\]

\subsection{Part I: The Proportional Derivative (PD) Controller}
The closed loop system under PD control takes the form
%
\begin{center}
  \begin{tikzpicture}[x=1in, y=1in]

    \node [draw, smooth_block] (Plant) {\(\frac{1}{s^2}\)};
    \node [draw, smooth_block, left = 0.50 of Plant] (Gain1) {\(K_d s + K_p\)};
    \node [draw, smooth_sum, left = 0.50 of Gain1] (Sum1) {};
    \node [right = 0.50 of Plant] (after_plant) {};
    \node [right = 0.50 of after_plant] (y) {};
    \node [left = 0.50 of Sum1] (r) {};

    \node [smooth_annotate, below = 0 of Gain1] {Controller};
    \node [smooth_annotate, below = 0 of Plant] {Plant};

    \draw [arrow, smooth_path]
      (Plant.east) -- (after_plant.base) -- (y.base) node [below right] {\(y\)};
    \draw [arrow, smooth_path]
      (Plant.east)
      --
      (after_plant.base)
      --
      +(0, -0.75)
      -|
      (Sum1.south)
      node [below right] {\(-\)};
    \draw [arrow, smooth_path]
      (Sum1.east)
      --
      (Gain1.west);
    \draw [arrow, smooth_path]
      (Gain1.east)
      --
      (Plant.west)
      node [below left] {\(u\)};
    \draw [arrow, smooth_path]
      (r.base)
      node [below left] {\(r\)}
      --
      (Sum1.west);
  \end{tikzpicture}
\end{center}
If you compute the closed loop transfer function \(G(s)\) from \(r(t)\) to \(y(t)\) you will find
\[
  G(s) = \frac{K_d s + K_p}{s^2 + K_d s + K_p}.
\]
As long as \(K_p, K_d > 0\) the closed loop system is stable.
Note that we have a zero appearing in the transfer function.
Assuming that the zero dynamics are fast, i.e. \(K_p \gg K_d,\) we can approximate the system by the transfer function
\[
  \tilde{G}(s) = \frac{K_p}{s^2 + K_d s + K_p}.
\]
It then should not surprise you that the intuition used for designing the gains in Lab~\ref{Lab:2} apply somewhat to the gains \(K_p\) and \(K_d.\)
You can see that decreasing (increasing) \(K_d\) while fixing \(K_p\) will result in a larger (smaller) settling time and larger (smaller) overshoot.
Decreasing (increasing) \(K_p\) while fixing \(K_d\) will result in smaller (larger) overshoot.
Of course, a real PID controller also has a non-zero integrator which can, at times, throw a wrench in this sort of analysis.
Your deliverable for this section is to
%
\begin{deliverable}[label={del:lab4:p1:1}]
  \textbf{Choose} and \textbf{record} a pair of gains \(K_p\) and \(K_d.\)
\end{deliverable}
%
and next procedure describes how to go about this.
%
\begin{procedure}[label={proc:lab4:kpkd}]
  In this procedure you will determine a pair \(K_p\) and \(K_d.\)
  \begin{enumerate}[label={(\arabic*)}]
    \item{%
      \textbf{Ensure} the loop of the Simulink model is closed as in the block diagram above.
    }
    \item{%
      \textbf{Pick} a random \(K_p\) and \(K_d\) in the slider range given in the Simulink model.
      \label{proc:lab4:kpkd:2}
    }
    \item{%
      \textbf{Run} the ``visualize\_uav.m'' MATLAB script and \textbf{inspect} Figure 2.
      If there are persistent oscillations in your response or if they don't dissipate around \SI{30}{s}, \textbf{repeat} step~\ref{proc:lab4:kpkd:2} with a different choice.
      Use your understanding of Lab~\ref{Lab:3} to motivate whether you should increase or decrease \(K_d\) or \(K_p.\)
    }
    \item{%
      Once you have settled on a choice for \(K_p\) and \(K_d,\)
      \textbf{run} the ``visualize\_uav.m'' MATLAB script and \textbf{inspect} Figure 2.
      \textbf{Observe} that there is a constant steady-state error.
    }
  \end{enumerate}
\end{procedure}

\subsection{Part II: Introducing an Integral Component}
We would like to eliminate the observed constant steady-state error.
How do we do this?
First we must determine why this error appears.
It is fair to ask:
``If our plant already has an integrator, how is it that there is constant steady-state error?''
In this simulation there is a constant \emph{input} disturbance similar to the \(d(t)\) of Lab~\ref{Lab:3}.
As you verified in Lab~\ref{Lab:3}, even an integrator in the plant cannot perfectly reject this input disturbance and can only attenuate it under proportional error feedback.
If we introduce a pole at the origin, i.e. an integrator, in the controller we can perfectly reject this error.
Unfortunately, there is a constraint on how large the integrator term can be.
Your deliverable for this part is to simply choose an integrator gain.
%
\begin{deliverable}[label={del:lab4:p2:1}]
  \textbf{Choose} and \textbf{record} a gain \(K_i > 0\) that perfectly rejects the step disturbance while ensuring closed loop stability.
\end{deliverable}
%
To do this we will make use of the root locus method.
As such, you are required to also produce a relevant root locus.
%
\begin{deliverable}[label={del:lab4:p2:2}]
  \textbf{Acquire} the root locus generated by the ``part\_2.m'' MATLAB script.
  \textbf{Place} a cursor on a point where the system is unstable so that you recognize how large \(K_i\) is allowed to be.
  \textbf{Save} the figure.
\end{deliverable}
%
Finally you will acquire the complete system response.
%
\begin{deliverable}[label={del:lab4:p2:3}]
  \textbf{Acquire} Figure 2 by running the ``visualize\_uav.m'' MATLAB script and \textbf{save} it.
\end{deliverable}
%
Your complete PID control system looks like
%
\begin{center}
  \begin{tikzpicture}[x=1in, y=1in]

    \node [draw, smooth_block] (Plant) {%
      \(\frac{1}{s^2}\)%
    };
    \node [draw, smooth_sum, left = 0.25 of Plant] (Sum2) {};
    \node [draw, smooth_block, left = 0.25 of Sum2] (Gain1) {%
      \(K_p + K_d s\)%
    };
    \node [draw, smooth_block, above = 0.25 of Gain1] (Gain2) {%
      \(\frac{K_i}{s}\)%
    };
    \node [draw, smooth_sum, left = 0.50 of Gain1] (Sum1) {};
    \node [right = 0.75 of Plant] (after_plant) {};
    \node [right = 0.75 of after_plant] (y) {};
    \node [left = 0.75 of Sum1] (r) {};

    \draw [arrow, smooth_path]
      (Plant.east) -- (after_plant.base) -- (y.base) node [below right] {\(y\)};
    \draw [arrow, smooth_path]
      (Plant.east)
      --
      (after_plant.base)
      --
      +(0, -0.75)
      -|
      (Sum1.south)
      node [below right] {\(-\)};
    \draw [arrow, smooth_path]
      (Sum1.east)
      --
      (Gain1.west);
    \draw [arrow, smooth_path]
      (Sum1.east)
      --
      +(0.25, 0)
      |-
      (Gain2.west);
    \draw [arrow, smooth_path]
      (Gain1.east)
      --
      (Sum2.west);
    \draw [arrow, smooth_path]
      (Gain2.east)
      -|
      (Sum2.north);
    \draw [arrow, smooth_path]
      (Sum2.east)
      --
      (Plant.west)
      node [below left] {\(u\)};
    \draw [arrow, smooth_path]
      (r.base)
      node [below left] {\(r\)}
      --
      (Sum1.west);
  \end{tikzpicture}
\end{center}
%
where you have determined \(K_p\) and \(K_d\) already.
The closed loop poles of this system\footnote{Notice what I do not say. I am \textbf{not} saying the closed loop zeros or the DC gain are the same.} are the same as the poles of the closed loop system
%
\begin{center}
  \begin{tikzpicture}[x=1in, y=1in]

    \node [draw, smooth_block] (Plant) {%
      \(\frac{1}{s(s^2 + K_d s + K_p)}\)%
    };
    \node [draw, smooth_block, left = 0.50 of Plant] (Gain1) {%
      \(K_i\)%
    };
    \node [draw, smooth_sum, left = 0.50 of Gain1] (Sum1) {};
    \node [right = 0.50 of Plant] (after_plant) {};
    \node [right = 0.50 of after_plant] (y) {};
    \node [left = 0.50 of Sum1] (r) {};

    \draw [arrow, smooth_path]
      (Plant.east) -- (after_plant.base) -- (y.base);
    \draw [arrow, smooth_path]
      (Plant.east)
      --
      (after_plant.base)
      --
      +(0, -0.50)
      -|
      (Sum1.south)
      node [below right] {\(-\)};
    \draw [arrow, smooth_path]
      (Sum1.east)
      --
      (Gain1.west);
    \draw [arrow, smooth_path]
      (Gain1.east)
      --
      (Plant.west);
    \draw [arrow, smooth_path]
      (r.base)
      --
      (Sum1.west);
  \end{tikzpicture}
\end{center}
It follows that if we plot the (positive gain) root locus of \(\frac{1}{s(s^2 + K_d s + K_p)}\) and choose a gain so that the poles of this system meet our desired specifications, we can use this very same gain as \(K_i\) in our closed loop system to produce the same poles that meet the same specifications.
The next procedure guides you through these steps.
%
\begin{procedure}[label={proc:lab4:ki}]
  In this procedure you will produce a root locus and make an initial choice for the gain \(K_i.\)
  \begin{enumerate}[label={(\arabic*)}]
    \item{%
      \textbf{Run} the ``part\_2.m'' MATLAB script with the variables \texttt{Kp} and \texttt{Kd} set to your chosen gains from Part I and \textbf{inspect} Figure 3.
    }
    \item{%
      \textbf{Observe} that if \(K_i\) is too large, the system will be rendered unstable. \textbf{Place a cursor} on one such point on the root locus.
      \textbf{Save} this Figure for Deliverable~\ref{del:lab4:p2:2}.
    }
    \item{%
      \textbf{Take note} of the gain depicted by your cursor in the previous step.
      If \(K_i\) takes on a value larger than this gain, your system may be rendered unstable!
    }
    \item{%
      \textbf{Pick any} value of \(K_i\) between \(0\) and the maximum gain defined by the previous step.
    }
    \item{%
      \textbf{Read} the cursor to find out what the gain value is.
      \textbf{Record} this gain value as the value you choose for \(K_i\) producing Deliverable~\ref{del:lab4:p2:1}.
    }
  \end{enumerate}
\end{procedure}
%
Now that you have chosen \(K_i\) you are ready to test out whether it worked.
There is no deliverable for the next procedure as it is only a verification that the previous procedures went smoothly.
%
\begin{procedure}[label={proc:lab4:verify}]
  You will now verify your choice of \(K_i.\)
  \begin{enumerate}[label={(\arabic*)}]
    \item{%
      \textbf{Set} the integrator gain with your chosen value of \(K_i.\)
    }
    \item{%
      \textbf{Run} the ``visualize\_uav.m'' MATLAB script and \textbf{inspect} Figures 1 and 2.
      \textbf{Observe} that the UAV continues to track the path but now does so without steady-state error.
      \textbf{Save} Figure 2 for Deliverable~\ref{del:lab4:p2:3}.
    }
  \end{enumerate}
\end{procedure}

\subsection{Part III: Tune your Controller}
Now that we have eliminated the steady-state error, you may observe more extreme oscillations or a large overshoot.
In this part, you will make an attempt to improve the performance of your controller.
Your deliverable for this section is
%
\begin{deliverable}[label={del:lab4:p3:1}]
  \textbf{Acquire} the root locus plots generated by the ``part\_3.m'' MATLAB script with your PID gains determined at the end of Part II.
  \textbf{Save} the figures.
\end{deliverable}
%
and you will use this information, alongside your other intuitions, to guide another choice of gains.
%
\begin{deliverable}[label={del:lab4:p3:2}]
  \textbf{Determine} a new set of gains \(K_p,\) \(K_i\) and \(K_d\) that in some way improve the performance of your UAV.
  This may be in reduced overshoot, or reduced settling time, for example.
  You may consider any characteristic you consider undesireable.
\end{deliverable}
%
Finally, you will acquire and save Figure 2 generated by ``visualize\_uav.m'' to prove you have achieved perfect tracking.
%
\begin{deliverable}[label={del:lab4:p3:3}]
  \textbf{Acquire} Figure 2 by running the ``visualize\_uav.m'' MATLAB script and \textbf{save} it to depict the improved performance.
  If there is no improvement, \textbf{acquire} atleast three simulations (of three different gain choices) producing Figure 2 and \textbf{save all of them}. \textbf{Record} the changes you made.
\end{deliverable}
%
\begin{procedure}[label={proc:lab4:tune}]
  There is no explicit procedure to tune your controller but this high-level series of steps can help you make your initial changes.
  \begin{enumerate}[label={(\arabic*)}]
    \item{%
      \textbf{Run} the ``part\_3.m'' MATLAB script with the variables \texttt{Kp}, \texttt{Kd} and \texttt{Ki} set to your gains determined in Part II.
      \textbf{Save} the figures (4, 5, 6) for Deliverable~\ref{del:lab4:p3:1}.
    }
    \item{%
      \textbf{Inspect} the root locus plots generated.
      Each root locus depicts your current closed loop poles with filled black circles.
      You can therefore see how changing any gain would change the location of the closed loop poles and thereby change the behaviour of your final closed loop system.
      The settling time of your system is determined primarily by the real part of the pole closest to the imaginary axis.
      The overshoot is determined primarily by the largest angle of the poles relative to the negative real axis.
      You can also see exactly what a pole will contribute to your system by clicking on the root locus to produce a cursor.
    }
    \item{%
      Guided by the location of the poles currently and the root locus plots produced, \textbf{decide} on a gain to change to improve atleast one characteristic of your system.
      \textbf{Make} the change to that gain. 
    }
    \item{%
      \textbf{Acquire} Figure 2 by running the ``visualize\_uav.m'' MATLAB script and \textbf{verify} that you did improve the characteristic you set out to improve.
      If successful, \textbf{save} it for Deliverable~\ref{del:lab4:p3:3} and \textbf{record} your gain choices for Deliverable~\ref{del:lab4:p3:2}.
      If not successful explore the range of parameters available to see what changes you can make happen.
      \emph{If no improvement is possible, save three simulations (Figure 2, in particular) of \textbf{atleast three distinct choices you trialed}. \textbf{Record} what those choices were.}
    }
  \end{enumerate}
\end{procedure}

\section{Report Deliverable}\label{Lab:4:Report}
Another lab nearly complete!
As usual, you are expected to submit a report demonstrating that you completed the lab and that you understand the tasks performed.
In addition to including
\begin{itemize}
  \item{Deliverable~\ref{del:lab4:p1:1},}
  \item{Deliverable~\ref{del:lab4:p2:1},}
  \item{Deliverable~\ref{del:lab4:p2:2},}
  \item{Deliverable~\ref{del:lab4:p2:3},}
  \item{Deliverable~\ref{del:lab4:p3:1},}
  \item{Deliverable~\ref{del:lab4:p3:2} and}
  \item{Deliverable~\ref{del:lab4:p3:3}}
\end{itemize}
in your report, you are required to answer the questions of the following deliverable.
Make sure to leverage your other deliverables in your answers!
\begin{deliverable}[label={lab4:report}]
  \begin{enumerate}[label={(\arabic*)}]
    \item{%
      Like in Lab~\ref{Lab:3}, the primary output was a (relative) position \(y(t).\)
      Unlike Lab~\ref{Lab:3}, we do not observe the velocity explicitly but, instead, use the differentiator in the PID controller to estimate velocity (derivative of the relative position is velocity) and apply that in closed loop feedback.
      Unfortunately, the explicit use of a differentiator in parallel to the proportional error feedback introduced zeros in the final transfer function \(G(s)\) from \(r(t)\) to \(y(t).\)
      In particular
      \[
        G(s) = K_d \frac{s^2 + \frac{K_p}{K_d} s + \frac{K_i}{K_d}}{s^3 + K_d s^2 + K_p s + K_i}.
      \]
      For your final choice of \(K_p,\) \(K_i\) and \(K_d,\) \textbf{determine} where the zeros of \(G(s)\) are located.
      \label{lab4:report:q1}
    }
    \item{%
      If the zeros are sufficiently fast (i.e. large negative real part) compared to the poles, we can approximate the behaviour of the system as simply the system where the zeros are in steady-state.
      Using the zeros you determined in~\ref{lab4:report:q1}, \textbf{determine} if the zeros affected your system's response.
      \textbf{Where} in the response of Figure 2 would it affect your system?
      You may assume that ``sufficiently fast'' means that the zeros is atleast one order of magnitude farther away in real part than any of the poles.
      \label{lab4:report:q2}
    }
    \item{%
      The PD controller of Part I produced a steady state constant error.
      This was explained, in Part II, as being caused by a step input disturbance to the plant \(P(s).\)
      \textbf{Prove} that if \(K_i \neq 0\) then step input disturbances to the plant are perfectly rejected.
      To do so, first find the transfer function from \(d(t)\) to \(y(t)\) in the diagram
      %
      \begin{center}
        \begin{tikzpicture}[x=1in, y=1in]

          \node [draw, smooth_block] (Plant) {\(\frac{1}{s^2}\)};
          \node [draw, smooth_sum, left = 0.50 of Plant] (Sum2) {};
          \node [above = 0.25 of Sum2] (d) {};
          \node [draw, smooth_block, left = 0.50 of Sum2] (Gain1) {\(\frac{K_d s^2 + K_p s + K_i}{s}\)};
          \node [draw, smooth_sum, left = 0.50 of Gain1] (Sum1) {};
          \node [right = 0.50 of Plant] (after_plant) {};
          \node [right = 0.50 of after_plant] (y) {};

          \draw [arrow, smooth_path]
            (Plant.east) -- (after_plant.base) -- (y.base) node [below right] {\(y\)};
          \draw [arrow, smooth_path]
            (Plant.east)
            --
            (after_plant.base)
            --
            +(0, -0.40)
            -|
            (Sum1.south)
            node [below right] {\(-\)};
          \draw [arrow, smooth_path]
            (Sum1.east)
            --
            (Gain1.west);
          \draw [arrow, smooth_path]
            (Gain1.east)
            --
            (Sum2.west);
          \draw [arrow, smooth_path]
            (Sum2.east)
            --
            (Plant.west)
            node [below left] {\(u\)};
          \draw [arrow, smooth_path]
            (d)
            node[right] {\(d(t)\)}
            --
            (Sum2.north);
        \end{tikzpicture}
      \end{center}
      %
      and use this transfer function to show that the output \(y(t)\) under the step input \(d(t) = A \mathbf{1}(t),\) for \(A \in \Real,\) converges to zero.
      \label{lab4:report:q3}
    }
    \item{%
      Using the root locuses of Deliverable~\ref{del:lab4:p3:1}, \textbf{explain} how the gains \(K_p,\) \(K_i\) and \(K_d\) theoretically affect the performance of your system as you vary them.
      Ensure you discuss each gain as well as discuss if those gains affect the qualities you discuss universally (does increasing a certain gain always result in the quality improving?).
      \label{lab4:report:q4}
    }
    \item{%
      The engineers at VenX claim that there are no input disturbances.
      Input disturbances, in this context, are caused by poor mechanical trimming of the aircraft (it doesn't ``naturally'' fly straight).
      Suppose you agreed with this statement\footnote{at the end of the day, you aren't a mechanical/aerospace engineer}.
      However, they argue that you should include the integrator term \emph{anyway} since they claim it improves the performance of your controller by guaranteeing that you converge onto the path (no steady-state error).

      Using the statement you proved in~\ref{lab4:report:q3}, any other relevant deliverables, your experience in this lab and the definition of \(P(s),\) \textbf{determine} if it is true that you should keep the integrator term and \textbf{explain} your reasoning. Ensure to explain both the positives and negatives of including an integrator term.
      \label{lab4:report:q5}
    }
    \item{%
      Repeat question~\ref{lab4:report:q5} except assuming that the plant was
      \[
        P(s) = \frac{s + 1}{s^2 + 2 s + 3}.
      \]
      Would this change your answer?
      \label{lab4:report:q6}
    }
    \item{%
      \textbf{State} what characteristic you aimed to improve in Part III.
      \textbf{Why} did you choose to change the gains you changed?
      A theoretical justification is not required but you should explain yourself clearly.
      If no change improved the characteristic, explain what changes you tried, why you tried them, and why you think they did not work.
      \label{lab4:report:q7}
    }
  \end{enumerate}
\end{deliverable}

\subsection{Grading Scheme}
The grading scheme is shown in Table~\ref{tab:lab4:grading}. The breakdown of
your grade is shown per deliverable except in the case of the lab
questions where it is shown per question.
%
\begin{table}
\centering
\begin{tabular}{c|l|c}
        & Deliverable           & Marks  \\ \hline
        & \ref{del:lab4:p1:1}         & 4       \\ \hline
        & \ref{del:lab4:p2:1}         & 4       \\ \hline
        & \ref{del:lab4:p2:2}         & 4      \\ \hline
        & \ref{del:lab4:p2:3}         & 4      \\ \hline
        & \ref{del:lab4:p3:1}         & 6       \\ \hline
        & \ref{del:lab4:p3:2}         & 4       \\ \hline
        & \ref{del:lab4:p3:3}         & 4       \\ \hhline{=|=|=}
Lab Subtotal&                       & 30      \\ \hhline{=|=|=}
        & \ref{lab4:report}~\ref{lab4:report:q1}  & 2       \\ \hline
        & \ref{lab4:report}~\ref{lab4:report:q2}  & 2       \\ \hline
        & \ref{lab4:report}~\ref{lab4:report:q3}  & 4       \\ \hline
        & \ref{lab4:report}~\ref{lab4:report:q4}  & 6       \\ \hline
        & \ref{lab4:report}~\ref{lab4:report:q5}  & 2      \\ \hline
        & \ref{lab4:report}~\ref{lab4:report:q6}  & 2      \\ \hline
        & \ref{lab4:report}~\ref{lab4:report:q7}  & 2      \\ \hhline{=|=|=}
Report Subtotal&  & 20 \\ \hhline{=|=|=}
  Total &                       & 50
\end{tabular}
\caption[Grading Scheme for Lab 4]{Grading scheme for Lab 4.}
\label{tab:lab4:grading}
\end{table}
%
