\chapter{Lead and Lag Compensator Design}\label{Lab:5}
Now that you have 
This time, you are designing an unmanned aerial vehicle that is dropped into the atmosphere and performs powered glides along predefined paths over the surface.
%
\begin{center}
  %\centering
  \input{Lab_4_UAV_PF.pdf_tex}%
  %\caption{Unmanned Aerial Vehicle (UAV) moving with respect to some line.}
  %\label{fig:lab4:uav}
\end{center}
%
The purpose is to do surveillance\footnote{Don't worry. The CEO of VenX, Lone Dusk, has assured you that the FBI are not using your technology for surveilling citizens in protests. Should you trust him? Probably not. But in this lab, you live in a utopia where people and the government are very honourable.} of the planet's surface to map out the landscape.

The controller you design controls the turning rate (\SI{}{rad/s}) and the output you measure is the distance to the straight line path \(y(t)\) (\SI{}{m}).
In practice the transfer function from this input to \(y(t)\) is highly nonlinear, but the senior engineers have designed an inner loop controller that makes the plant \(P(s)\) look like
\[
  P(s) = \frac{1}{s^2}.
\]
Take \texttt{ECE 486}\footnote{In the robotics course you will learn to just cancel the nonlinearities out.} if you are interested in how to do this.

 --- a very similar system to the one you created in Lab~\ref{Lab:4}.
In this lab, you do not have access to the transfer function for \(P(s)\) directly.
Instead you will acquire the frequency response (Bode plot) empirically via MATLAB and perform your design against this.
You will also verify closed loop stability using an empirically acquired Nyquist plot.
This is quite common in practice for plants that are difficult to model or for plants where it is hard to perform parameter estimation due to the large number of parameters and sensitivity.

\section{Objectives}\label{Lab:4:Objectives}
This goals of this lab are to
\begin{enumerate}[label=(\arabic*)]
  \item{
    \textbf{Practice} the Lead and Lag Design Procedures as you've learned them in your course.
  }
  \item{
    \textbf{Learn} how delays in the loop can affect stability.
  }
  \item{
    \textbf{Learn} how a phase margin specification determines the maximum delay \(L(s)\) can afford before losing stability.
  }
\end{enumerate}
The deliverable dependency graph is
\begin{center}
\begin{tikzpicture}[x=1em, y=1em]
  \node[deliverable] (D1) {Deliverable\\\ref{del:lab4:p1:1}};
  \node[deliverable, below = 2 of D1] (D2) {Deliverable\\\ref{del:lab4:p2:1}--\ref{del:lab4:p2:3}};
  \node[deliverable, below = 2 of D2] (D3) {Deliverable\\\ref{del:lab4:p3:1}--\ref{del:lab4:p3:3}};
  \node[deliverable, right = 2 of D3] (Q1) {Deliverable\\\ref{lab4:report}~\ref{lab4:report:q1}--\ref{lab4:report:q2}};
  \node[deliverable, below = 2 of Q1] (Q4) {Deliverable\\\ref{lab4:report}~\ref{lab4:report:q4}};
  \node[deliverable, right = 2 of Q4] (Q5) {Deliverable\\\ref{lab4:report}~\ref{lab4:report:q5}};
  \node[deliverable, at = {(Q5|-D1)}] (Q3) {Deliverable\\\ref{lab4:report}~\ref{lab4:report:q3}};
  \node[deliverable, right = 2 of Q5] (Q6) {Deliverable\\\ref{lab4:report}~\ref{lab4:report:q6}};
  \node[deliverable, left = 2 of D3] (Q7) {Deliverable\\\ref{lab4:report}~\ref{lab4:report:q7}};

  \draw[signal, arrow] (D1.south) -- (D2.north);
  \draw[signal, arrow] (D2.south) -- (D3.north);
  \draw[signal, arrow] (D3.east) -- (Q1.west);
  \draw[signal, arrow] (D3.south) |- (Q4.west);
  \draw[signal, arrow] (Q4.east) -- (Q5.west);
  \draw[signal, arrow] (Q3.south) -- (Q5.north);
  \draw[signal, arrow] (Q5.east) -- (Q6.west);
  \draw[signal, arrow] (D3.west) -- (Q7.east);

\end{tikzpicture}
\end{center}

\section{Experimental Procedure}\label{Lab:4:Experiment}
This lab is split into four parts.
In Part I you will simulate 
In Part II you will analyze how the integral term affects the performance of your system.
Finally in Part III you make an incremental change to your PID controller to improve your system's performance qualitatively.
The plant for this lab will be
\[
  P(s) = \frac{1}{s^2}.
\]

\subsection{Preamble: Delays}
A ``perfect'' delay by time \(\tau\) from an input signal \(u(t)\) to an output signal \(y(t)\) is determined by the static equation
\[
  y(t) = u(t - \tau).
\]
Verify for yourself that this definition ensures the value of the output \(y\) at some time \(t\) is given by the value of the input signal \(u\) at a time \(\tau\) units prior.
Take the Laplace transform of both sides\footnote{and suppose the integrals converge} to find that
\[
  Y(s) = e^{-\tau s} U(s).
\]
It is in this sense that one can view \(e^{-\tau s}\) as the transfer function of the delay operator.
This is significant as it gives a frequency-domain version of the time delay allowing us to translate a time-domain specification to a frequency-domain specification.
%
Suppose a loop transfer function \(L(s)\) satisfies the Nyquist criterion.
Equivalently, imagine that the closed loop system
%
\begin{center}
  \begin{tikzpicture}[x=1in, y=1in]
    \node [draw, smooth_block] (Plant) {\(L(s)\)};
    \node [draw, smooth_block, below = 0.25 of Plant] (Delay) {\(1\)};
    \node [draw, smooth_sum, left = 0.50 of Plant] (Sum1) {};
    \node [right = 0.50 of Plant] (after_plant) {};
    \node [right = 0.50 of after_plant] (y) {};
    \node [left = 0.50 of Sum1] (r) {};

    \draw [arrow, smooth_path]
      (Plant.east) -- (after_plant.base) -- (y.base) node [below right] {\(y\)};
    \draw [arrow, smooth_path]
      (Plant.east)
      --
      (after_plant.base)
      |-
      (Delay.east);
    \draw [arrow, smooth_path]
      (Delay.west)
      -|
      (Sum1.south)
      node [below right] {\(-\)};
    \draw [arrow, smooth_path]
      (Sum1.east)
      --
      (Plant.west);
    \draw [arrow, smooth_path]
      (r.base)
      node [below left] {\(r\)}
      --
      (Sum1.west);
  \end{tikzpicture}
\end{center}
%
is stable.
We would like to meet the specification that:
\begin{quote}
  The closed loop system remains stable under a delay of \(\tau\) seconds.
\end{quote}
In terms of transfer functions we can imagine asking that the closed loop system 
%
\begin{center}
  \begin{tikzpicture}[x=1in, y=1in]
    \node [draw, smooth_block] (Plant) {\(L(s)\)};
    \node [draw, smooth_block, below = 0.25 of Plant] (Delay) {\(e^{-\tau s}\)};
    \node [draw, smooth_sum, left = 0.50 of Plant] (Sum1) {};
    \node [right = 0.50 of Plant] (after_plant) {};
    \node [right = 0.50 of after_plant] (y) {};
    \node [left = 0.50 of Sum1] (r) {};

    \draw [arrow, smooth_path]
      (Plant.east) -- (after_plant.base) -- (y.base) node [below right] {\(y\)};
    \draw [arrow, smooth_path]
      (Plant.east)
      --
      (after_plant.base)
      |-
      (Delay.east);
    \draw [arrow, smooth_path]
      (Delay.west)
      -|
      (Sum1.south)
      node [below right] {\(-\)};
    \draw [arrow, smooth_path]
      (Sum1.east)
      --
      (Plant.west);
    \draw [arrow, smooth_path]
      (r.base)
      node [below left] {\(r\)}
      --
      (Sum1.west);
  \end{tikzpicture}
\end{center}
%
If you compare the Nyquist plot of \(L(s)\) with the Nyquist plot of \(e^{-\tau s} L(s)\) you will observe that they look very similar;
in fact, a number of points of the latter Nyquist plot look like a rotated version of the former Nyquist plot.
This notion (if put formally) allows us to translate the earlier delay specification into the following phase margin specification: 
\begin{quote}
  The loop transfer function \(L(s)\) has a phase margin strictly greater than \(\tau \omega_{\mathrm{gc}}\) where \(\omega_{\mathrm{gc}}\) is the gain crossover frequency in \SI{}{\radian}.
\end{quote}
This is the primary specification you will meet.
Note that this translation of specification only makes sense when \(L(s)\) is already stable.
Also note that changes in the phase margin could change the gain crossover and thereby change what delays \(L(s)\) can afford.

\subsection{Part I: Delays and Stability}
Consider the simple proportional feedback loop
%
\begin{center}
  \begin{tikzpicture}[x=1in, y=1in]

    \node [draw, smooth_block] (Plant) {\(P(s)\)};
    \node [draw, smooth_block, left = 0.50 of Plant] (Gain1) {\(K\)};
    \node [draw, smooth_sum, left = 0.50 of Gain1] (Sum1) {};
    \node [right = 0.50 of Plant] (after_plant) {};
    \node [right = 0.50 of after_plant] (y) {};
    \node [left = 0.50 of Sum1] (r) {};

    \node [smooth_annotate, below = 0 of Gain1] {Controller};
    \node [smooth_annotate, below = 0 of Plant] {Plant};

    \draw [arrow, smooth_path]
      (Plant.east) -- (after_plant.base) -- (y.base) node [below right] {\(y\)};
    \draw [arrow, smooth_path]
      (Plant.east)
      --
      (after_plant.base)
      --
      +(0, -0.75)
      -|
      (Sum1.south)
      node [below right] {\(-\)};
    \draw [arrow, smooth_path]
      (Sum1.east)
      node [above right] {\(e\)}
      --
      (Gain1.west);
    \draw [arrow, smooth_path]
      (Gain1.east)
      --
      (Plant.west);
    \draw [arrow, smooth_path]
      (r.base)
      node [below left] {\(r\)}
      --
      (Sum1.west);
  \end{tikzpicture}
\end{center}
where \(P(s)\) is the dynamics of line-following drone.
Your first task is to
%
\begin{deliverable}[label={del:lab5:p1:1}]
  \textbf{Choose} a loop gain \(0 < K < 1000\) so that the steady-state step tracking error is less than or equal to \(1\%.\)
\end{deliverable}
%
You will then acquire a Nyquist plot to verify stability of the closed loop system when there are no delays as well as acquire a Nyquist plot to demonstrate instability when the delay is introduced.
%
\begin{deliverable}[label={del:lab5:p1:2}]
  \textbf{Acquire} two Nyquist plots (using the Model Linearizer app) so that
  \begin{itemize}
    \item{one demonstrates the stability of the gain-compensated closed loop system when there is no delay and}
    \item{the other demonstrates the instability of the gain-compensated closed loop system with a delay.}
  \end{itemize}
\end{deliverable}
%
Finally, you will need to acquire a Bode plot of the undelayed open loop system in order to perform the Lead and Lag design procedures.
%
\begin{deliverable}[label={del:lab5:p1:3}]
  \textbf{Acquire} a Bode plot (using the Model Linearizer app) of the undelayed open loop system.
\end{deliverable}
%
To complete these tasks, follow the next procedure.
%
\begin{procedure}[label={proc:lab5:1}]
  In this procedure you will determine a gain \(K\) and acquire the required Nyquist plots.
  You will also get a chance to simulate the system.
  \begin{enumerate}[label={(\arabic*)}]
    \item{%
      Assume the plant \(P(s)\) has a DC gain \(1.\)
      The transfer function from \(r(t)\) to \(e(t)\) is
      \[
        \frac{1}{1 + K P(s)}.
      \]
      \textbf{Choose} a gain \(K\) so that the DC gain of this transfer function is less than \(0.01.\)
      This assures a steady-state tracking error of less than \(1\%.\)
    }
    \item{%
      \textbf{Change} the variable \texttt{K} \emph{in the MATLAB script} ``procedure\_5\_1.m'' to match your chosen gain.
      \textbf{Run} the script.
    }
    \item{%
      \textbf{Open} the ``Lab\_5\_Model\_System.slx'' Simulink model and \textbf{ensure} that the switch is pointed towards the path without the ``Delay.''
    }
    \item{%
      The model should already be correctly configured with the ``Open Loop Input'' set to the signal \(e(t)\) and the ``Open Loop Output'' set to signal \(y(t).\)
      \textbf{Verify} this.
    }
    \item{%
      \textbf{Open} the Model Linearizer App and \textbf{acquire} a Nyquist plot.
      This produces part of Deliverable~\ref{del:lab5:p1:2}.
      Assuming \(P(s)\) has no unstable poles, \textbf{verify} that you have closed loop stability.
    }
    \item{%
      \textbf{Open} the Model Linearizer App and \textbf{acquire} a Bode plot.
      \textbf{Identify} the \SI{0}{\decibel} gain crossover and the phase margin.
      This completes Deliverable~\ref{del:lab5:p1:3}.
    }
    \item{%
      \textbf{Open} the ``Lab\_5\_Model\_System.slx'' Simulink model and \textbf{ensure} that the switch is pointed towards the path \textbf{with} the ``Delay.''
    }
    \item{%
      \textbf{Open} the Model Linearizer App and \textbf{acquire} a Nyquist plot.
      This completes Deliverable~\ref{del:lab5:p1:2}.
      Assuming \(P(s)\) has no unstable poles, \textbf{verify} that you do not have closed loop stability.
    }
  \end{enumerate}
\end{procedure}

\subsection{Part II: Design a Lag controller}
One way to increase the phase margin of a loop transfer function is to introduce a Lag controller.
Lag controllers decrease the gain so as to change the gain crossover;
changing the gain crossover in turn changes the phase margin.
In this part the closed loop system takes the form
%
\begin{center}
  \begin{tikzpicture}[x=1in, y=1in]

    \node [draw, smooth_block] (Plant) {\(K P(s)\)};
    \node [draw, smooth_block, left = 0.50 of Plant] (Gain1) {\(C_\mathrm{lag}(s)\)};
    \node [draw, smooth_sum, left = 0.50 of Gain1] (Sum1) {};
    \node [right = 0.50 of Plant] (after_plant) {};
    \node [right = 0.50 of after_plant] (y) {};
    \node [left = 0.50 of Sum1] (r) {};

    \node [smooth_annotate, below = 0 of Gain1] {Controller};
    \node [smooth_annotate, below = 0 of Plant] {Plant};

    \draw [arrow, smooth_path]
      (Plant.east) -- (after_plant.base) -- (y.base) node [below right] {\(y\)};
    \draw [arrow, smooth_path]
      (Plant.east)
      --
      (after_plant.base)
      --
      +(0, -0.75)
      -|
      (Sum1.south)
      node [below right] {\(-\)};
    \draw [arrow, smooth_path]
      (Sum1.east)
      node [above right] {\(e\)}
      --
      (Gain1.west);
    \draw [arrow, smooth_path]
      (Gain1.east)
      --
      (Plant.west);
    \draw [arrow, smooth_path]
      (r.base)
      node [below left] {\(r\)}
      --
      (Sum1.west);
  \end{tikzpicture}
\end{center}
%
where
\[
  C_\mathrm{lag}(s)
    =
      \frac{\alpha T s + 1}{T s + 1}
  ,
  \;
  0 < \alpha < 1
  ,
  \;
  T > 0.
\]
is a Lag compensator with unit DC gain.
Your task is to find \(\alpha\) and \(T\) in terms of a phase margin specification and the Bode plot of \(K P(s)\) --- determined by Deliverable~\ref{del:lab5:p1:3} --- to complete the Lag control design.
%
\begin{deliverable}[label={del:lab5:p2:1}]
  Let \(\omega_\mathrm{gc}\) determine the \SI{0}{\decibel} gain crossover frequency (in \SI{}{\radian}) of \(K P(s).\) 
  This can determined by looking at the Bode plot from Deliverable~\ref{del:lab5:p1:3}.
  \textbf{Design} a Lag compensator (determine \(\alpha\) and \(T\)) so that the Bode plot of the compensated open loop system has a phase margin strictly greater than \(\frac{6\omega_\mathrm{gc}}{100}~\mathrm{rad}.\)
\end{deliverable}
%
You will then acquire plots that show that the delayed system has been stabilized with the Lag compensator.
%
\begin{deliverable}[label={del:lab5:p2:2}]
  \textbf{Acquire} a Nyquist plot demonstrating the Lag compensated system \emph{with delay} is closed loop stable.
\end{deliverable}
%
\begin{procedure}[label={proc:lab5:2}]
  In this procedure you will design a Lag compensator, using the content you learned in your course, and simulate the delayed closed loop system.
  \begin{enumerate}[label={(\arabic*)}]
    \item{%
      Using the \SI{0}{\decibel} gain crossover frequency \(\omega_\mathrm{gc}\) found for \(K P(s)\) in Deliverable~\ref{del:lab5:p1:3} the engineers at VenX have determined that compensated closed loop system must have a phase margin strictly greater than
      \[
        \frac{6\omega_\mathrm{gc}}{100}~\mathrm{rad}.
      \]
      Using the Bode plot in Deliverable~\ref{del:lab5:p1:3}, \textbf{determine} the required gain crossover to achieve this specification.
    }
    \item{%
      \textbf{Choose} \(\alpha\) and \(T\) in a way to meet this specification.
      This produces Deliverable~\ref{del:lab5:p2:1}.
    }
    \item{%
      \textbf{Change} the variable \texttt{alpha} and \texttt{T} \emph{in the MATLAB script} ``procedure\_5\_2.m'' to match your chosen parameters.
      \textbf{Run} the script.
    }
    \item{%
      \textbf{Open} the ``Lab\_5\_Model\_System.slx'' Simulink model and \textbf{ensure} that the switch is pointed towards the path \textbf{with} the ``Delay.''
    }
    \item{%
      \textbf{Open} the Model Linearizer App and \textbf{acquire} a Nyquist plot.
      This completes Deliverable~\ref{del:lab5:p2:2}.
      Assuming \(P(s)\) has no unstable poles, \textbf{verify} that you do not have closed loop stability.
    }
  \end{enumerate}
\end{procedure}

\subsection{Part III: Design a Lead controller}
Another way to increase the phase margin of a loop transfer function is to introduce a Lead controller.
Lead controllers add phase at the crossover frequency.
In this part the closed loop system takes the form
%
\begin{center}
  \begin{tikzpicture}[x=1in, y=1in]

    \node [draw, smooth_block] (Plant) {\(K P(s)\)};
    \node [draw, smooth_block, left = 0.50 of Plant] (Gain1) {\(C_\mathrm{lead}(s)\)};
    \node [draw, smooth_sum, left = 0.50 of Gain1] (Sum1) {};
    \node [right = 0.50 of Plant] (after_plant) {};
    \node [right = 0.50 of after_plant] (y) {};
    \node [left = 0.50 of Sum1] (r) {};

    \node [smooth_annotate, below = 0 of Gain1] {Controller};
    \node [smooth_annotate, below = 0 of Plant] {Plant};

    \draw [arrow, smooth_path]
      (Plant.east) -- (after_plant.base) -- (y.base) node [below right] {\(y\)};
    \draw [arrow, smooth_path]
      (Plant.east)
      --
      (after_plant.base)
      --
      +(0, -0.75)
      -|
      (Sum1.south)
      node [below right] {\(-\)};
    \draw [arrow, smooth_path]
      (Sum1.east)
      node [above right] {\(e\)}
      --
      (Gain1.west);
    \draw [arrow, smooth_path]
      (Gain1.east)
      --
      (Plant.west);
    \draw [arrow, smooth_path]
      (r.base)
      node [below left] {\(r\)}
      --
      (Sum1.west);
  \end{tikzpicture}
\end{center}
%
where
\[
  C_\mathrm{lead}(s)
    =
      \frac{\alpha T s + 1}{T s + 1}
  ,
  \;
  \alpha > 1
  ,
  \;
  T > 0.
\]
is a Lead compensator with unit DC gain.
Your task is to find \(\alpha\) and \(T\) in terms of a phase margin specification and the Bode plot of \(K P(s)\) --- determined by Deliverable~\ref{del:lab5:p1:3} --- to complete the Lead control design.
%
\begin{deliverable}[label={del:lab5:p3:1}]
  Let \(\omega_\mathrm{gc}\) determine the \SI{0}{\decibel} gain crossover frequency (in \SI{}{\radian}) of \(K P(s).\) 
  This can determined by looking at the Bode plot from Deliverable~\ref{del:lab5:p1:3}.
  \textbf{Design} a Lead compensator (determine \(\alpha\) and \(T\)) so that the Bode plot of the compensated open loop system has a phase margin strictly greater than \(\frac{6\omega_\mathrm{gc}}{100}~\mathrm{rad}.\)
\end{deliverable}
%
You will then acquire plots that show that the delayed system has been stabilized with the Lead compensator.
%
\begin{deliverable}[label={del:lab5:p3:2}]
  \textbf{Acquire} a Nyquist plot demonstrating the Lead compensated system \emph{with delay} is closed loop stable.
\end{deliverable}
%
\begin{procedure}[label={proc:lab5:3}]
  In this procedure you will design a Lag compensator, using the content you learned in your course, and simulate the delayed closed loop system.
  \begin{enumerate}[label={(\arabic*)}]
    \item{%
      Using the \SI{0}{\decibel} gain crossover frequency \(\omega_\mathrm{gc}\) found for \(K P(s)\) in Deliverable~\ref{del:lab5:p1:3} the engineers at VenX have determined that compensated closed loop system must have a phase margin strictly greater than
      \[
        \frac{6\omega_\mathrm{gc}}{100}~\mathrm{rad}.
      \]
      Using the Bode plot in Deliverable~\ref{del:lab5:p1:3}, \textbf{determine} the required phase addition to meet this specification.
    }
    \item{%
      \textbf{Choose} \(\alpha\) and \(T\) in a way to meet this specification.
      This produces Deliverable~\ref{del:lab5:p3:1}.
    }
    \item{%
      \textbf{Change} the variable \texttt{alpha} and \texttt{T} \emph{in the MATLAB script} ``procedure\_5\_3.m'' to match your chosen parameters.
      \textbf{Change} the gain \texttt{K} in that same script if your design procedure demands it.
      \textbf{Run} the script.
    }
    \item{%
      \textbf{Open} the ``Lab\_5\_Model\_System.slx'' Simulink model and \textbf{ensure} that the switch is pointed towards the path \textbf{with} the ``Delay.''
    }
    \item{%
      \textbf{Open} the Model Linearizer App and \textbf{acquire} a Nyquist plot.
      This completes Deliverable~\ref{del:lab5:p3:2}.
      Assuming \(P(s)\) has no unstable poles, \textbf{verify} that you do not have closed loop stability.
    }
  \end{enumerate}
\end{procedure}

\subsection{Part IV: Simulating the Real System}
In Parts II and III you designed Lag and Lead compensators that ensured the closed loop system was stable even under a delay;
this was verified using the Nyquist plots of Deliverable~\ref{del:lab5:p2:2} and~\ref{del:lab5:p3:2}.
In this part, you will simulate the real closed loop system with delays.
Consider the closed loop system
%
\begin{center}
  \begin{tikzpicture}[x=1in, y=1in]

    \node [draw, smooth_block] (Plant) {\(P(s)\)};
    \node [draw, smooth_block, left = 0.40 of Plant] (delay1) {%
      \(e^{-\frac{\tau}{2} s}\)
    };
    \node [draw, smooth_block, right = 0.40 of Plant] (delay2) {%
      \(e^{-\frac{\tau}{2} s}\)
    };
    \node [draw, smooth_block, left = 0.40 of delay1] (Gain1) {%
      \(K\frac{\alpha T s + 1}{T s + 1}\)%
    };
    \node [draw, smooth_sum, left = 0.40 of Gain1] (Sum1) {};
    \node [right = 0.40 of delay2] (after_plant) {};
    \node [right = 0.40 of after_plant] (y) {};
    \node [left = 0.40 of Sum1] (r) {};

    \node [smooth_annotate, below = 0 of Gain1] {Controller};
    \node [smooth_annotate, below = 0 of Plant] {Plant};
    \node [smooth_annotate, below = 0 of delay1] {Send Delay};
    \node [smooth_annotate, below = 0 of delay2] {Receive Delay};

    \draw [arrow, smooth_path]
      (Plant.east)
      --
      (delay2.west);
    \draw [arrow, smooth_path]
      (delay2.east) -- (after_plant.base) -- (y.base) node [below right] {\(y\)};
    \draw [arrow, smooth_path]
      (delay2.east)
      --
      (after_plant.base)
      --
      +(0, -0.75)
      -|
      (Sum1.south)
      node [below right] {\(-\)};
    \draw [arrow, smooth_path]
      (Sum1.east)
      node [above right] {\(e\)}
      --
      (Gain1.west);
    \draw [arrow, smooth_path]
      (Gain1.east)
      --
      (delay1.west);
    \draw [arrow, smooth_path]
      (delay1.east)
      --
      (Plant.west);
    \draw [arrow, smooth_path]
      (r.base)
      node [below left] {\(r\)}
      --
      (Sum1.west);
  \end{tikzpicture}
\end{center}
%
The ``Send Delay'' captures the amount of time the plant has to wait to see an update from the controller on the ground while the ``Receive Delay'' captures the amount of time the controller has to wait to see an updated output.
From a networking perspective, this is lower bounded by the ping time\footnote{lower bounded since it ignores any computational delays!}.
If you designed your Lag and Lead compensators correctly in Parts II and III, then you should expect them to perform adequately.
Remember, stability is the primary characteristic of concern in this discussion.
Do not be alarmed if the settling time or overshoot is undesireable.

There are precisely two deliverables for this part.
First you will simulate the real system against the Lag compensator to produce
%
\begin{deliverable}[label={del:lab5:p4:1}]
  \textbf{Acquire} Figure 542 generated by ``procedure\_5\_4\_simulate.m.''
\end{deliverable}
%
Second, you will simulate the real system against the Lead compensator to produce
%
\begin{deliverable}[label={del:lab5:p4:2}]
  \textbf{Acquire} Figure 552 generated by ``procedure\_5\_5\_simulate.m.''
\end{deliverable}
%
To do so, follow the next couple procedures.
%
\begin{procedure}[label={proc:lab5:4}]
  In this procedure you will simulate the real closed loop system with your Lag compensator.
  \begin{enumerate}[label={(\arabic*)}]
    \item{%
      \textbf{Run} ``procedure\_5\_4\_simulate.m.''
    }
    \item{%
      \textbf{Save} Figure 542.
      \textbf{Verify} the simulatation ran to completion (\SI{240}{\second}) and that the error settles.
      If there are oscillations, they should decay.
      This produces Deliverable~\ref{del:lab5:p4:1}.
    }
  \end{enumerate}
\end{procedure}
%
\begin{procedure}[label={proc:lab5:5}]
  In this procedure you will simulate the real closed loop system with your Lead compensator.
  \begin{enumerate}[label={(\arabic*)}]
    \item{%
      \textbf{Run} ``procedure\_5\_5\_simulate.m.''
    }
    \item{%
      \textbf{Save} Figure 552.
      \textbf{Verify} the simulatation ran to completion (\SI{240}{\second}) and that the error settles.
      If there are oscillations, they should decay.
      This produces Deliverable~\ref{del:lab5:p4:2}.
    }
  \end{enumerate}
\end{procedure}


\section{Report Deliverable}\label{Lab:5:Report}
You are almost done the lab component of this course!
As usual, you are expected to submit a report demonstrating that you completed the lab and that you understand the tasks performed.
In addition to including
\begin{itemize}
  \item{Deliverable~\ref{del:lab5:p1:1},}
  \item{Deliverable~\ref{del:lab5:p1:2},}
  \item{Deliverable~\ref{del:lab5:p1:3},}
  \item{Deliverable~\ref{del:lab5:p2:1},}
  \item{Deliverable~\ref{del:lab5:p2:2},}
  \item{Deliverable~\ref{del:lab5:p3:1},}
  \item{Deliverable~\ref{del:lab5:p3:2},}
  \item{Deliverable~\ref{del:lab5:p4:1},}
  \item{Deliverable~\ref{del:lab5:p4:2},}
\end{itemize}
in your report, you are required to answer the questions of the following deliverable.
Make sure to leverage your other deliverables in your answers!
\begin{deliverable}[label={lab5:report}]
  \begin{enumerate}[label={(\arabic*)}]
    \item{%
      \label{lab5:report:q1}
    }
    \item{%
      \label{lab5:report:q2}
    }
    \item{%
      \label{lab5:report:q3}
    }
    \item{%
      \label{lab5:report:q4}
    }
    \item{%
      \label{lab5:report:q5}
    }
    \item{%
      \label{lab5:report:q6}
    }
    \item{%
      \label{lab5:report:q7}
    }
  \end{enumerate}
\end{deliverable}

\subsection{Grading Scheme}
The grading scheme is shown in Table~\ref{tab:lab5:grading}. The breakdown of
your grade is shown per deliverable except in the case of the lab
questions where it is shown per question.
%
\begin{table}
\centering
\begin{tabular}{c|l|c}
        & Deliverable           & Marks  \\ \hline
        & \ref{del:lab5:p1:1}         & 4       \\ \hline
        & \ref{del:lab5:p1:2}         & 4       \\ \hline
        & \ref{del:lab5:p1:3}         & 4      \\ \hline
        & \ref{del:lab5:p2:1}         & 4      \\ \hline
        & \ref{del:lab5:p2:2}         & 4       \\ \hline
        & \ref{del:lab5:p3:1}         & 4       \\ \hline
        & \ref{del:lab5:p3:2}         & 4       \\ \hline
        & \ref{del:lab5:p4:1}         & 4       \\ \hline
        & \ref{del:lab5:p4:2}         & 4       \\ \hhline{=|=|=}
Lab Subtotal&                       & 36      \\ \hhline{=|=|=}
        & \ref{lab5:report}~\ref{lab5:report:q1}  & 2       \\ \hline
        & \ref{lab5:report}~\ref{lab5:report:q2}  & 2       \\ \hline
        & \ref{lab5:report}~\ref{lab5:report:q3}  & 2       \\ \hline
        & \ref{lab5:report}~\ref{lab5:report:q4}  & 2      \\ \hline
        & \ref{lab5:report}~\ref{lab5:report:q5}  & 2      \\ \hline
        & \ref{lab5:report}~\ref{lab5:report:q6}  & 2      \\ \hline
        & \ref{lab5:report}~\ref{lab5:report:q7}  & 2      \\ \hhline{=|=|=}
Report Subtotal&  & 14 \\ \hhline{=|=|=}
  Total &                       & 50
\end{tabular}
\caption[Grading Scheme for Lab 5]{Grading scheme for Lab 5.}
\label{tab:lab5:grading}
\end{table}
%
